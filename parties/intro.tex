\chapter*{Introduction}
\addcontentsline{toc}{chapter}{Introduction}
\markboth{Introduction}{Introduction}
\label{chap:introduction}
\vspace{5mm}
%Pourquoi privilégier le NoSQL au SQL dans le domaine des SIG et comment l'appliquer ?
\paragraph{}75\% des applications que nous utilisons ont intégré la géolocalisation \supercite{statGeolocalisation} : de la simple cartographie aux services de transports en passant par les jeux et les réseaux sociaux, tous utilisent cette technologie. Cette fonctionnalité, parfois critiquée de par la vente des données récoltées, peut se révéler utile voire indispensable dans certaines situations. 

\paragraph{}Derrière cet outil se trouve un ensemble plus important : les\newacronym{SIG}{SIG}{Systèmes d'Information Géographiques} \gls{SIG}. Ils sont conçus pour recueillir, stocker, traiter, analyser, gérer mais également présenter les données spatiales et géographiques.

\paragraph{}La première apparition de l'analyse de données géographiques remonte à la première moitié du \textsc{xix}\ieme ~siècle dans le cadre de l'analyse d'une épidémie au sein d'un département français. Cependant, la vraie avancée dans ce domaine a eu lieu lors de la deuxième partie du \textsc{xx}\ieme ~siècle par la mise en place des premiers SIG. Depuis, la démocratisation des ordinateurs et d'Internet a permis son expansion.

\paragraph{}Les systèmes d'information sont omniprésents dans notre environnement. Qu'ils soient géographiques ou non, ils comportent un ensemble de données les permettant de fonctionner. Selon les besoins, les \newacronym{SGBD}{SGBD}{systèmes de gestion de base de données} \gls{SGBD} peuvent être relationnels ou s'éloigner de ce paradigme en utilisant du NoSQL.

\paragraph{}Pour ce mémoire, j'ai souhaité travailler autour des données, domaine qui m'intéresse surtout avec l'expension du NoSQL. Du fait qu'elles sont utilisées dans de nombreux domaines, j'ai souhaité l'affilier à un autre domaine qui m'intéresse : la géographie.

\paragraph{}Les SIG possèdent des données spatiales complétées par d'autres alphanumériques. Leur stockage est réalisé de façon à les équilibrer. Pour que ces données stockées soient utilisées de manière optimale, elles se doivent d'être modélisées auparavant. Afin de répondre à cela et aux besoins, les modèles n'ont cessé d'évoluer. Avant les années 1970, les modèles orientés texte (ressemblant aux fichiers CSV), hiérarchique et réseau étaient utilisés. Ensuite est apparu le modèle relationnel qui est aujourd'hui le plus communément utilisé à la fois dans les entrepôts de données mais également dans une partie des systèmes d'information. Ce modèle se définit par le fait qu'un ensemble de tuples ayant les mêmes attributs est appelé « relation ». Depuis, de nouveaux modèles, s'éloignant de ce paradigme, ont vu le jour avec les bases de données NoSQL : orientés objet, document, colonne ou graphe \supercite{modeles}. Selon les outils les utilisant, la modélisation des données et le modèle utilisé diffère selon les besoins.


\paragraph{}Afin de traiter le sujet, une étude sera faite sur le domaine des \acrshort{SIG}, les outils les utilisant et les modèles associés. A partir de cela, une étude comparative, basé sur plusieurs critères permettra de définir quel paradigme et quel modèle est le plus adapté pour manipuler des données géogragraphiques.
%Afin de traiter le sujet, une étude sera faite sur les différents structures et modèles utilisés par les différents outils que nous connaissons. A partir de cela, nous pourrons établir les critères permettant de choisir le modèle de données le plus adéquate que nous appliquerons à un exemple.



